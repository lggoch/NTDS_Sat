\documentclass[11pt,letterpaper]{article}
%\topmargin -.45in
\textwidth 6.5in
\textheight 9.in
\oddsidemargin 0in
\headheight 0in
\usepackage{graphicx}
\usepackage{fancybox}
%\usepackage{palatino}
\usepackage[utf8]{inputenc} %solucion del problema de los acentos.
\usepackage{epsfig,graphicx}
\usepackage{multicol,pst-plot}
\usepackage{pstricks}
\usepackage{amsmath}
\usepackage{amsfonts}
\usepackage{amssymb}
\usepackage{eucal}
\usepackage[left=2cm,right=2cm,top=2cm,bottom=2cm]{geometry}
\pagestyle{empty}
\DeclareMathOperator{\tr}{Tr}
\newcommand*{\op}[1]{\check{\mathbf#1}}
\newcommand{\bra}[1]{\langle #1 |}
\newcommand{\ket}[1]{| #1 \rangle}
\newcommand{\braket}[2]{\langle #1 | #2 \rangle}
\newcommand{\mean}[1]{\langle #1 \rangle}
\newcommand{\opvec}[1]{\check{\vec #1}}
\renewcommand{\sp}[1]{$${\begin{split}#1\end{split}}$$}

\usepackage{lipsum}

\usepackage{listings}
\usepackage{color}

\definecolor{codegreen}{rgb}{0,0.6,0}
\definecolor{codegray}{rgb}{0.5,0.5,0.5}
\definecolor{codepurple}{rgb}{0.58,0,0.82}
\definecolor{backcolour}{rgb}{0.95,0.95,0.92}

\lstdefinestyle{mystyle}{backgroundcolor=\color{backcolour},   
    commentstyle=\color{codegreen},
    keywordstyle=\color{magenta},
    numberstyle=\tiny\color{codegray},
    stringstyle=\color{codepurple},
    basicstyle=\footnotesize,
    breakatwhitespace=false,         
    breaklines=true,                 
    captionpos=b,                    
    keepspaces=true,                 
    numbers=left,                    
    numbersep=5pt,                  
    showspaces=false,                
    showstringspaces=false,
    showtabs=false,                  
    tabsize=2
}

\lstset{style=mystyle}

\begin{document}
\pagestyle{plain}
\begin{flushleft}
\underline{Ecole Polytechnique Federale de Lausanne} \hfill
\end{flushleft}

\begin{flushright}\vspace{-5mm}
\includegraphics[height=1.5cm]{logo.png}
\end{flushright}
 
\begin{center}\vspace{-1cm}
\textbf{\large A Network tour of Data Science}\\
Project Proposal Update\\
A.Weber, L.Gauchoux, L.Loiselle
\end{center}

 
\rule{\linewidth}{0.1mm}
%%%%%%%%%%%%%%%%%%%%%%%%%%%%%%%%%%%%%%%%%%%%%%%%%%%%%%%%%%%%%%%%%%%%%%%%

\section{Introduction}

This document provide an update plan to the project proposal previously written

\section{Data Acquisition and Exploration}

The dataset comes from the website "space-track.org".
To extract the data, we needed to create an account, but this is free and easy to do.

We will use TLE information for the main telecommunication satellite arrays in our project, which are GLOBALSTAR, ORBCOMM, IRIDIUM, INMARSAT and INTELSAT.

We might also use historical satellite data, that we had previously gathered.


\section{Data Exploitation}

From the telecommunication TLE, we plan to characterise the satellites by their owner, from their orbital parameters.

To do this, we will first need the location of each satellites.

For geostationnary satellites, the location is easy to find, because its position relative to earth doesn't change, this is the case of the INTELSAT and INMARSAT.

For other satellite arrays, it is more complex, because their satellites are in Low Earth Orbit and will have different position depending on the time of measurement of the TLE\@.
To measure their position, we will use data on their orbits and suppose that they weren't corrected during recent time, i.e.\ they kept the same orbit.
With this supposition, we will be able to extract their geographical location at a specific time since the epoch and have a basis for comparison.

Once we have the positions of each satellites, we will create a graph of distance between satellites. 
Also, we might include more information in the graph such as the orbital parameters to help in our characterisation.

Another information that we might add to the graph is the land area seen by the satellite, because it is an important characteristic of coverage.

With this network of graph, we will attempt to subdivide the satellites between the different companies.
This task is different than what we did in class, because high density of satellite will likely mean that the satellite are not from the same company.
Indeed, satellite constellations are built to have maximum ground coverage, which implies that they attempt to be uniformly distributed to maximise this coverage.

This will be the main task of the project.

Once we have done this task, we could expand our research to other aspect of the telecommunication satellite industry.
For example, we have data on the debris produced by each company and that are still in orbit.
We could show the evolution of those debris with time and maybe use some of the network generation tool that we have seen in class to reproduce the historical production of debris.
If we can find a precise match, we could then project the debris production into the future and generate new information from the dataset.

Finally, the methods used for the project could also be used to show a satellite coverage on a map. 
This could be interesting to give a temporal estimation of the number of satellite above our head at any given time.



\end{document}


